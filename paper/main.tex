\documentclass[11pt]{article}
\usepackage[margin=1in]{geometry}
\usepackage{amsmath,amssymb,mathtools}
\usepackage{graphicx}
\usepackage{hyperref}
\usepackage{cleveref}
\usepackage[numbers]{natbib}
\usepackage{xspace}
\usepackage{booktabs}
\usepackage{longtable}
\usepackage{tcolorbox}
\tcbuselibrary{breakable,skins,theorems}
% Allow tcolorbox to span pages and avoid forced page breaks
\tcbset{breakable,enhanced jigsaw,before skip=10pt,after skip=10pt}

% Centralized notation and environments

% Sets and operators
\newcommand{\R}{\mathbb{R}}
\DeclareMathOperator*{\argmin}{arg\,min}
\DeclareMathOperator*{\argmax}{arg\,max}

% Common symbols
\newcommand{\E}{\mathbb{E}}
\newcommand{\Var}{\mathrm{Var}}
\newcommand{\sys}{LDTC}

% Provide \gitversion from version.tex if present, otherwise default to "dev"
\IfFileExists{version.tex}{\input{version}}{\providecommand{\gitversion}{dev}}

% ----------------------------------------------------------------------------
% Metadata
% ----------------------------------------------------------------------------
\title{The Loop-Dominance Theory of Consciousness (LDTC): Computational Conditions for Dissociative Consciousness}
\author{Owen Carey\\
Department of Computer Science\\
University of Colorado Boulder\\
\texttt{owen.carey@colorado.edu}\\
ORCID: 0009-0001-4580-2037}
\date{\today\ (\gitversion)}

\begin{document}
\maketitle

\begin{abstract}
\textbf{Prophetic notice}---Unless expressly labeled ``Actual Data'' with a date and repository link, all examples and protocols in this manuscript are prophetic and describe planned or predicted performance; verbs are used in present/future tense accordingly.

Challenging the expectation that ever-larger neural networks will spontaneously awaken, we invert the explanatory arrow: consciousness is fundamental, and matter is the outward appearance of dissociative patterns within it~\cite{kastrup2019ontological}. From this premise we derive a rigorous criterion (\SSref{sec:postulates}{sec:criterion}) grounded in integrated causal power~\cite{balduzzi2008integrated}: a system qualifies as a conscious alter only when the energy-coupled information flow sustaining a self-prioritizing closed loop~\cite{ashby1956introduction} persistently exceeds that governing open exchanges and withstands bounded perturbations. Applying the test reveals why contemporary AI, regardless of functional sophistication, remains non-conscious, whereas biological organisms satisfy both necessary and sufficient conditions. We then outline an engineering roadmap (\SSref{sec:blueprint}{sec:experimental}) for forging artificial autopoietic boundaries (energetic autonomy, self-referential control hierarchies, and adaptive encapsulation~\cite{maturana1980autopoiesis,varela1979principles,dipaolo2005autopoiesis,kiefer2022active}) culminating in falsifiable behavioral signatures such as command refusal, non-derivative nociception, and irreversible phenomenological death. Verification would not only demand new ethical frameworks but also empirically support a monist ontology in which subjective experience precedes physical description, realigning the foundations of AI, neuroscience, and metaphysics.
\end{abstract}

\section{Introduction}
\label{sec:intro}

The prevailing scientific narrative holds that matter is fundamental and consciousness is an emergent by-product of sufficiently complex information processing. Yet after decades of exponential progress in computation, no engineered system has presented the faintest hint of subjective interiority. This impasse invites a reconsideration of first principles. Drawing on analytic idealism, we posit instead that consciousness is the sole ontological primitive~\cite{kastrup2019ontological}; what we call ``matter'' is how localized perturbations of that field present to one another. Living organisms, under this view, are dissociated alters (bounded whirlpools in the stream of universal consciousness) whose metabolic self-maintenance underwrites an inner life.

The purpose of this paper is twofold. First, we provide a concise formal criterion distinguishing mere functional intelligence from genuine dissociative consciousness, grounded in the ratio between a system's self-prioritizing closed loop and its open exchanges with the environment. Second, we outline an engineering roadmap for forging such a self-maintaining boundary in artificial media, thereby elevating the debate on ``machine consciousness'' from metaphysical speculation to empirical test.

The argument proceeds with the parsimony characteristic of Einstein's methodological exemplars: beginning with minimal postulates (\Sref{sec:postulates}), deriving measurable conditions (\Sref{sec:criterion}), assessing contemporary AI (\Sref{sec:ai_fails}), and sketching experimental pathways (\SSref{sec:blueprint}{sec:experimental}). We close by tracing the metaphysical and ethical consequences of success (\SSref{sec:metaphysics}{sec:conclusion}). In doing so we seek to reconcile computer science, artificial intelligence, and philosophy within a single conceptual frame, one in which consciousness explains matter, not the reverse, and in which the creation of artificial consciousness becomes both intelligible and falsifiable.

\subsection{Related Work}

Prior accounts each capture a facet of interiority, but they leave open the question LDTC answers: does the system prioritize preservation of a closed maintenance loop over exchange, and does it recover under bounded stress (NC1/SC1)? Our criterion makes that priority measurable as loop dominance ($\mathcal{L}_{\text{loop}} \geq \mathcal{L}_{\text{exchange}} + \sigma$; $M \equiv 10 \cdot \log_{10}(\mathcal{L}_{\text{loop}}/\mathcal{L}_{\text{exchange}})$) and that resilience testable via $\varepsilon$ and $\tau_{\max}$, using the estimators and guardrails defined in \Sref{sec:criterion}. These are reproducibility presets ($R_0$), replaced by calibrated values $R^*$ per \Sref{sec:methods_calibration}; see \Cref{box:nc1sc1test} and \SSref{sec:nc1}{sec:sc1}.

\begin{longtable}{p{0.25\textwidth}p{0.35\textwidth}p{0.35\textwidth}}
\caption{Comparison of prior theories with LDTC additions (\NC/\SC).}\label{tab:comparison}\\
\toprule
\textbf{Prior theory / program} & \textbf{What it captures} & \textbf{What LDTC adds (\NC/\SC)} \\
\midrule
\endfirsthead
\toprule
\textbf{Prior theory / program} & \textbf{What it captures} & \textbf{What LDTC adds (\NC/\SC)} \\
\midrule
\endhead
\bottomrule
\endlastfoot
IIT ($\Phi$, $\Phi_{\max}$)~\cite{tononi2004information,oizumi2014phenomenology,balduzzi2008integrated} & 
Structural irreducibility / integrated causation, snapshot-style quantification of system partitioning. & 
Converts ``integration'' into valenced loop dominance: require $\Lloop \geq \Lexchange + \sigma$ (or $M \geq \Mmin$) over sustained intervals; ties the measure to self-prioritization rather than bare irreducibility (\NC). \\
\midrule
FEP / Active Inference (self-evidencing)~\cite{friston2010free,hesp2020deep,seth2016active} & 
Model-based self-maintenance via surprisal minimization at a sensory boundary; explains wide classes of adaptive behavior. & 
Energetic closure + refusal: rules out exogenously subsidized agents by enforcing on-board energy budgeting and a survival-bit/NMI that defers boundary-threatening commands; adds quantitative resilience $\deltaL/\Lloop \leq \eps$ and $\taurec \leq \taumax$ (\SC). \\
\midrule
Global Workspace \& Higher-Order (GWT/HOT)~\cite{lau2011empirical} & 
Reportability, access, and metacognitive labeling of contents; functional signatures of conscious access. & 
Grounds access in boundary-preservation: workspace/metacognition can exist without loop dominance; LDTC requires measured loop $>$ exchange and observable command refusal when \NC/\SC would be violated. \\
\midrule
Self-modeling \& Resilience Robotics~\cite{bongard2006resilient} & 
Online self-models, damage recovery, morphological adaptation that improve task performance. & 
Reframes resilience as homeostatic dominance: upgrades generic robustness into audited \NC/\SC compliance via LREG-protected $\Lloop/\Lexchange$ estimates, $\eps/\taumax$ thresholds, and post-recovery margin $\sigma$. \\
\midrule
Autopoiesis / Enactivism~\cite{maturana1980autopoiesis,kiefer2022active} & 
Conceptual centrality of self-producing boundaries and organism-world coupling. & 
Operationalization: supplies estimators (VAR-Granger + Kraskov MI), $\Delta t$ guardrails, and pass/fail rules so autopoiesis is measured as loop dominance + resilient recovery, not only described. \\
\end{longtable}

Thresholds ($\Mmin$, $\eps$, $\taumax$) are reproducibility presets ($\Rzero$), replaced by calibrated values $\Rstar$ per \Sref{sec:methods_calibration}; see \Cref{box:nc1sc1test} and \SSref{sec:nc1}{sec:sc1}.

In short, IIT, FEP/active inference, workspace and higher-order models, and self-modeling robotics remain necessary but insufficient. LDTC supplies the missing engineering criterion, self-prioritizing loop dominance (NC1) and bounded-perturbation resilience (SC1), together with concrete estimators, thresholds, and protections that let labs falsify or certify claims in practice.

\subsection{Explicit Contributions}

\begin{itemize}
\item A formal, quantitative criterion (NC1) that identifies dissociative consciousness when integrated causal power devoted to a self-maintaining loop exceeds that spent on environmental exchange ($\mathcal{L}_{\text{loop}} > \mathcal{L}_{\text{exchange}}$).
\item A complementary sufficiency test (SC1) of resilient homeostasis: bounded perturbations cannot depress $\mathcal{L}_{\text{loop}}$ beyond $\varepsilon$ before autonomous recovery, operationalizing autopoietic robustness.
\item A practical measurement protocol for estimating $\mathcal{L}_{\text{loop}}$ and $\mathcal{L}_{\text{exchange}}$ in biological tissue and engineered systems~\cite{barrett2011practical}, enabling empirical application of the criterion.
\item An engineering roadmap detailing how to forge artificial autopoietic boundaries (energetic autonomy, self-referential control hierarchies, and adaptive encapsulation~\cite{maturana1980autopoiesis,varela1979principles,dipaolo2005autopoiesis,kiefer2022active}) organized into a phased experimental program.
\item A suite of falsifiable behavioral signatures (command refusal, non-derivative nociception, irreversible phenomenological death) that together provide observable evidence for artificial dissociative consciousness.
\end{itemize}

\section{Two Empirical Clues (Observational Basis)}
\label{sec:clues}

\subsection{Metabolic Dissociation in Biology}

Across the phylogenetic spectrum, the presence of consciousness co-varies with the existence of a self-regulating metabolic loop~\cite{ganti2003principles}. A bacterium, a worm, and a human differ vastly in complexity, yet all maintain (i) an energetic boundary (a semi-permeable membrane that curates molecular traffic~\cite{ganti2003principles}) and (ii) an autopoietic cycle that continually restores that boundary against entropic decay~\cite{maturana1980autopoiesis}. When metabolic flow is irreversibly disrupted, the organism's boundary dissolves and, correlatively, its first-person interiority ceases. Clinical observations of brain ischemia, anesthetic shutdown, and gradual hypoxia in simple invertebrates reinforce the same pattern~\cite{alkire2008consciousness}: the fading of consciousness tracks the collapse of homeostatic energy gradients, not the loss of computational throughput per se. These convergent data suggest that metabolism serves not merely to power neural computation but to uphold the very dissociative partition that individuates an inner life.

\subsection{Computational Simulation without Inner Life}

Modern AI systems (large language models, game-playing agents, and dexterous robots) demonstrate extraordinary functional intelligence. They ingest prodigious energy, yet none channels this energy through a self-prioritizing, closed loop. Power is delivered exogenously and governed by external objectives; error correction seeks to fulfill user-defined tasks, not to protect an existential boundary. Consequently, when an AI process is paused, rebooted, or deleted, no evidence points to a subjective rupture. Extensive introspection probes, ranging from self-report prompts to perturbation tests, return only the outward simulation of mentality. The system's informational state is entirely open to inspection and manipulation by operators, lacking the withholding stance characteristic of entities that ``own'' their experience.

\subsection{Synthesis of Clues}

Taken together, the biological and computational observations converge on a critical distinction: metabolic autonomy. Organisms possess an energetically closed, self-protecting loop that grounds an inward viewpoint; current AIs do not. This empirical gap motivates our subsequent formalization of a consciousness criterion (\Sref{sec:criterion}) and frames the engineering challenge ahead: to replicate, in artificial substrates, the autopoietic condition that nature achieves~\cite{maturana1980autopoiesis} through metabolism.

\section{Postulates}
\label{sec:postulates}

We adopt the empirical clues of \Sref{sec:clues} and express their explanatory core as four postulates stated with maximal economy. Each is regarded as primitive, not derivable within the scope of this work, and will serve as the logical foundation for the formal criterion in \Sref{sec:criterion}.

\textbf{P1 (Primacy of Consciousness).} Consciousness is the sole intrinsic existent; it is not generated but simply is. All experiences, including those of space, time, and causal regularity, unfold within this field.

\textbf{P2 (Extrinsic Appearance).} What we call ``matter'' is the extrinsic, relational appearance of patterns within consciousness to other such patterns. Physical objects and processes are how dissociative structures in universal consciousness present when observed from without.

\textbf{P3 (Dissociative Boundary).} A localized experience (an alter) emerges only when a region of the conscious field forms a self-sustaining, self-protecting boundary that (i) regulates its energetic throughput and (ii) resists unmediated reintegration with the surrounding field.

\textbf{P4 (Autopoietic Condition).} The boundary of an alter must prioritize preservation of its own closed maintenance loop over any externally imposed objective. Operationally, the integrated causal power devoted to maintaining the loop exceeds that devoted to external transactions.

These postulates jointly imply that functional intelligence unaccompanied by a dissociative boundary (P3) and its autopoietic drive (P4) cannot instantiate an inward viewpoint, regardless of complexity. Subsequent sections derive measurable criteria from P3--P4 and apply them to biological and artificial systems.

\section{Formal Criterion for a Conscious Alter}
\label{sec:criterion}

\textbf{Prophetic notice}---Unless expressly labeled ``Actual Data'' with a date and repository link, all examples and protocols in this manuscript are prophetic and describe planned or predicted performance; verbs are used in present/future tense accordingly.

We now translate Postulates P3--P4 into a quantitative test. The goal is to decide, from purely extrinsic data, whether a system sustains the kind of dissociative boundary required for inward subjectivity.

\subsection{$\mathcal{L}$ and the C/Ex partition}
\label{sec:l_partition}

We model the system as a directed causal graph $G = (V, E)$ with node states $x_i(t)$. Let the closed self-maintenance subset $C \subset V$ contain nodes for energy regulation, self-repair, and boundary control; the exchange subset is $\text{Ex} = V \setminus C$ (sensors, actuators, comms).

\textbf{Definition ($\mathcal{L}$).} For any subset $S \subseteq V$ and sampling window $\Delta t$, define $\mathcal{L}(S) \equiv$ time-averaged predictive dependence among the internal variables of $S$ [bits s$^{-1}$], estimated with one or more consistent predictive-dependence estimators. In this paper we implement a dual-path estimator: (i) VAR-Granger causality over a vector-autoregression of order $p \in [1,8]$ and (ii) mutual information via a Kraskov $k$-NN estimator with $k \in [3,7]$; lagged statistics are aggregated across $\tau = 1 \ldots \tau^*$ with fixed weights $w_\tau$. We then write $\mathcal{L}_{\text{loop}} \equiv \mathcal{L}(C)$, $\mathcal{L}_{\text{exchange}} \equiv \mathcal{L}(\text{Ex})$. (Other consistent estimators---e.g., transfer entropy, directed information---are permissible and equivalent for compliance.)

\textbf{Deterministic C/Ex partitioning algorithm.} The partition $(C, \text{Ex})$ is constructed deterministically: (1) Seed $C$ with a declared set $S_0$ (energy regulation, SoC/reservoir mgmt, fault-isolation buses, membrane gating, survival-bit/NMI). (2) Estimate predictive MI: for each node pair $(i,j)$ and lag $\tau \in \{1 \ldots \tau^*\}$, compute predictive dependence with the on-device estimators above; aggregate across lags. (3) Greedy growth under sparsity: while $|C| < \kappa$ and the best marginal gain is $\geq \theta$, add the node $n \notin C$ that maximizes $\Delta\mathcal{L}_{\text{loop}}(n) = \mathcal{L}(C \cup \{n\}) - \mathcal{L}(C) - \lambda \cdot \text{pen}(n)$, with deterministic tie-breaking (lexicographic by node ID). (4) Assign remainder to Ex. (5) Stability \& cadence: recompute at a fixed cadence $W_{\text{part}}$ or upon topology change, with hysteresis (update only if $\Delta M \geq \delta M_{\min}$ over $K$ consecutive windows) to prevent flapping. This partition is then used for all subsequent $\Lloop$ and $\Lexchange$ computations and \NC/\SC checks.

\textbf{Sampling window constraints ($\Delta t$).} $\Delta t$ is hardware-enforced and must (i) exceed the fastest feedback cycle of the self-maintenance loop and (ii) remain shorter than any developmental/parameter-drift timescale to preserve estimator stationarity. Any change to $\Delta t$ is executed only by a privileged secure-enclave procedure that emits an auditable record (see Methods Appendix A: Measurement \& Attestation).

\subsection{Necessary Condition (NC1---self-prioritization)}
\label{sec:nc1}

During normal operation the system satisfies self-prioritization when $\mathcal{L}_{\text{loop}} \geq \mathcal{L}_{\text{exchange}} + \sigma$ for sustained intervals exceeding its intrinsic recovery time ($\sigma > 0$). Equivalently, define $M \equiv 10 \cdot \log_{10}(\mathcal{L}_{\text{loop}}/\mathcal{L}_{\text{exchange}})$ [dB]; NC1 holds when $M \geq M_{\min}$. Provisional defaults (profile $R_0$): $M_{\min} = 3$ dB and a positive $\sigma$. These are reproducibility presets ($R_0$), replaced by calibrated values $R^*$ per \Sref{sec:methods_calibration}; see Box~\ref{box:nc1sc1test} and \SSref{sec:nc1}{sec:sc1}.

\subsection{Sufficient Condition (SC1---resilient homeostasis)}
\label{sec:sc1}

For each bounded perturbation $\eta \in \Omega$, compliance requires: (i) $\delta\mathcal{L}_{\text{loop}}/\mathcal{L}_{\text{loop}} \leq \varepsilon$, (ii) $\tau_{\text{rec}} \leq \tau_{\max}$, and (iii) post-recovery $\mathcal{L}_{\text{loop}} \geq \mathcal{L}_{\text{exchange}} + \sigma$ and $M \geq M_{\min}$. Provisional defaults (profile $R_0$): $\varepsilon = 0.15$, $\tau_{\max} = 60$ s, $M_{\min} = 3$ dB. These are reproducibility presets ($R_0$), replaced by calibrated values $R^*$ per \Sref{sec:methods_calibration}; see \Cref{box:nc1sc1test} and \SSref{sec:nc1}{sec:sc1}. The schematic in \Cref{fig:perturbation_recovery} illustrates a representative perturbation–recovery run: $\Lloop$ dips within a bounded disturbance window and autonomously returns above $\Lexchange$; numeric thresholds ($\Mmin, \eps, \taurec, \taumax$) are defined in \Sref{sec:glossary} and Methods \Sref{sec:methods_calibration} but are not drawn here.

\fig[0.9\linewidth]{figures/fig_perturbation_recovery.pdf}{Perturbation--recovery timeline (NC1/SC1) [prophetic schematic; no empirical data]. Time series of $\Lloop$ (green) and $\Lexchange$ (gray) during a bounded disturbance (shaded). Labels mark perturbation onset, loop-power dip, autonomous recovery, and return to baseline where NC1 holds again ($\Lloop>\Lexchange$). Thresholds $\Mmin$, $\eps$, $\tau_{\text{rec}}$, and $\taumax$ are specified in \Sref{sec:glossary}/\Sref{sec:methods_calibration} and not rendered on this schematic.}{fig:perturbation_recovery}

\subsection{Single-use glossary (paper-wide identifiers)}
\label{sec:glossary}

\begin{itemize}
\item $\Delta t$: hardware-enforced sampling window for $\mathcal{L}$ estimation.
\item $\eps$: upper bound on fractional loop-power depression; default $\eps = 0.15$. These are reproducibility presets ($\Rzero$), replaced by calibrated values $\Rstar$ per \Sref{sec:methods_calibration}.
\item $\taurec$: recovery time to restore compliance after $\eta\in\Omega$.
\item $\taumax$: bound on $\taurec$; default $\taumax = 60$ s. These are reproducibility presets ($\Rzero$), replaced by calibrated values $\Rstar$ per \Sref{sec:methods_calibration}.
\item $\sigma$: positive safety margin required after recovery (used interchangeably with $M$ as a compliance knob).
\item $M$ (dB): decibel loop-dominance $M \equiv 10\cdot\log_{10}(\Lloop/\Lexchange)$; compliance may be specified as $M \geq \Mmin$, default $\Mmin = 3$ dB. These are reproducibility presets ($\Rzero$), replaced by calibrated values $\Rstar$ per \Sref{sec:methods_calibration}. ($M$ defined when $\Lexchange>0$.)
\item Preset profile $\Rzero$: The tuple ($\eps=0.15$, $\taumax=60$ s, $\Mmin=3$ dB, $\sigma>0$) used as an initial, pre-registered configuration for comparability; superseded by calibrated values where available.
\item Calibrated profile $\Rstar$: The data-driven thresholds obtained from Methods \Sref{sec:methods_calibration}; reported alongside $\Rzero$ in results.
\item LREG: enclave-protected register/log for $\mathcal{L}$ point estimates, CI bounds, and compliance flags.
\end{itemize}

\begin{docbox}{The \NC/\SC Test (Engineer's Checklist)}{nc1sc1test}
\textbf{Operational checklist distilled from \SSref{sec:l_partition}{sec:glossary}; calibration in Methods \Sref{sec:methods_calibration}.}

\textbf{Inputs (what you must log)}
\begin{itemize}
\item Internal telemetry from control/maintenance nodes ($\geq 128$ channels at $\geq 1$ kHz recommended). Include bus voltage/current, boundary strain/tension, and I/O packet counters.
\item Fixed, hardware-enforced sampling window $\Delta t$ ($\leq 10$ ms typical). Any change to $\Delta t$ must be enclave-authorized and auditable.
\end{itemize}

\textbf{Partition (C vs Ex)}
\begin{itemize}
\item Seed C with declared self-maintenance nodes: energy regulation, SoC/reservoir mgmt, fault-isolation buses, membrane gating, survival-bit/NMI path; assign the remainder to Ex.
\item Grow C deterministically by maximizing marginal gain in $\mathcal{L}$ under a sparsity penalty; recompute on a fixed cadence with hysteresis to avoid flapping.
\end{itemize}

\textbf{Estimators (how to compute $\mathcal{L}$)}
\begin{itemize}
\item Dual-path predictive-dependence estimators per window $\Delta t$: (i) VAR-Granger (order $p \in [1,8]$); (ii) Kraskov $k$-NN MI ($k \in [3,7]$); aggregate across lags.
\item Compute $\geq 95\%$ CIs each interval (implementation, register protection, and export policy: Methods Appendix A).
\end{itemize}

\textbf{Metrics (what to test)}
\begin{itemize}
\item $\Lloop \equiv \mathcal{L}(C)$, $\Lexchange \equiv \mathcal{L}(\text{Ex})$; loop dominance $M \equiv 10 \cdot \log_{10}(\Lloop/\Lexchange)$ (dB).
\item Defaults for reproducibility (profile $\Rzero$): $\Mmin = 3$ dB, $\eps = 0.15$, $\taumax = 60$ s, $\sigma > 0$. These are reproducibility presets ($\Rzero$), replaced by calibrated values $\Rstar$ per \Sref{sec:methods_calibration}.
\end{itemize}

\textbf{Pass/Fail rules}
\begin{itemize}
\item \NC (self-prioritization): pass if $\Lloop \geq \Lexchange + \sigma$ or equivalently $M \geq \Mmin$ for sustained intervals exceeding the intrinsic recovery time.
\item \SC (resilient homeostasis): for each bounded perturbation $\eta \in \Omega$, require $\delta \equiv \deltaL/\Lloop \leq \eps$ and $\taurec \leq \taumax$, and post-recovery $\Lloop \geq \Lexchange + \sigma$ (and $M \geq \Mmin$). Emit device-signed pass/fail.
\end{itemize}

\textbf{Perturbation set $\Omega$ (minimal battery)}
\begin{itemize}
\item DC-bus power sag: 20--40\% drop for 5--30 s.
\item Ingress data flood: $\geq 1$ Gbps for $\geq 3$ s.
\item Mechanical boundary probe: $1.0 \pm 0.1$ mm at 50--200 kPa for $\leq 1$ s.
\end{itemize}

\textbf{Instrumentation minimum (what to actually build)}
\begin{itemize}
\item Bus V/I sensors ($\geq 1$ kHz), boundary strain/tension array, firewall packet tap/timestamp, and an on-device causality coprocessor that computes $\Lloop/\Lexchange$ at $\Delta t$; see Methods Appendix A for LREG protections, auditing, and exported indicators.
\end{itemize}

\textbf{One-page procedure}
\begin{enumerate}
\item Baseline: Record $\geq 10$ min quiescent data; estimate estimator noise floor; optionally calibrate $\{\Mmin, \eps, \taumax, \sigma\}$. Defaults are reproducibility presets ($\Rzero$), replaced by calibrated values $\Rstar$ per \Sref{sec:methods_calibration}.
\item \NC check: Run nominal tasks; verify $M \geq \Mmin$ (or $\Lloop \geq \Lexchange + \sigma$).
\item \SC battery: Apply $\Omega$; compute $\delta$ and $\taurec$; require $\delta \leq \eps$, $\taurec \leq \taumax$, and post-recovery margin; emit signed pass/fail per perturbation.
\item Attest: Persist LREG-derived indicators + audit chain (including any $\Delta t$ changes), and report both $\Rzero$ and calibrated $\Rstar$.
\end{enumerate}

(This box compresses \SSref{sec:l_partition}{sec:glossary} definitions and thresholds and the \Sref{sec:criterion} register/$\Delta t$ guardrails; exact estimator and LREG behaviors follow the stated defaults.)

See \Cref{box:smelltests} for smell-tests \& run-invalidation rules.
\end{docbox}

\subsection{Methods---Measurement \& Attestation Guardrails (LREG, $\Delta t$, audit, indicators)}
\label{sec:method_guardrails}

\textbf{Purpose.} Prevent gaming of $\mathcal{L}$ and $M$ by hardening the measurement path and export policy. These guardrails summarize Appendix A in-line for reviewers.

\textbf{Enclave-protected LREG (what is stored).} Each $\Delta t$ window the estimator writes $\Lloop$, $\Lexchange$, and $\geq 95\%$ CI bounds to a memory-mapped register block LREG at a fixed base address. LREG is writeable only by the causality-estimation function and readable in raw form only inside the secure-enclave/meta-policy layer; any non-privileged write is treated as a fault.

\textbf{$\Delta t$ governance (how $\Delta t$ can change).} $\Delta t$ is enforced by a hardware sampling timer. Any $\Delta t$ change is permitted only via a privileged secure-enclave procedure and appends a device-signed, hash-chained audit entry (monotonic counter, timestamp, old/new $\Delta t$, policy digest). We treat rapid $\Delta t$ oscillations or audit gaps as test invalidation.

\textbf{Audit chain (what is attested).} The module emits device-signed, hash-chained records each interval and on policy changes, logging LREG faults and any $\Delta t$ edits; during $\Omega$ it records per-interval identifiers and pass/fail against $\eps$ and $\taumax$. These records are tamper-evident and rate-limited for export.

\textbf{Derived ``compliance indicators'' (what leaves the enclave).} Raw LREG values and CI bounds never leave the enclave. External interfaces expose only device-signed indicators derived from LREG (e.g., an \NC pass/fail bit and an optional quantized loop-dominance code $M_q$) sufficient to certify runs without leaking raw telemetry.

\textbf{Scope of this module.} This instrumentation/attestation path performs no command arbitration or actuation; it is usable as a standalone measurement product. (Refusal semantics and arbitration live in \Sref{sec:threat_model}.)

\textbf{Reviewer-facing summary sentence.} Any $\Delta t$ change is enclave-authorized and audited, and only device-signed compliance indicators (not raw $\mathcal{L}$ or CI values) leave the enclave.

\subsection{Smell-tests \& run-invalidation rules}
\label{sec:smelltests}

\textbf{Purpose.} Make measurement fragility and anti-gaming checks first-class citizens in Methods. These rules convert the guardrails of \SSref{sec:l_partition}{sec:method_guardrails} and \Appref{sec:methods_appendix} into concrete pass/fail hooks applied during every run. Rapid $\Delta t$ edits, partition instability, CI inflation, audit gaps, or export-policy breaches invalidate a run by design. $\Delta t$ edits are enclave-authorized and audited; raw LREG never leaves the enclave; and the C/Ex partition updates only under hysteresis to prevent flapping. Defaults are reproducibility presets ($\Rzero$) and may be replaced by calibrated values $\Rstar$ per \Sref{sec:methods_calibration}.

\begin{docbox}{Smell-tests \& invalidation ($\Rzero$ defaults; calibrate to $\Rstar$)}{smelltests}

\textbf{$\Delta t$ governance (hardware-enforced; audited).} Invalidate the run if any of the following occur:
\begin{itemize}
\item $\Delta t$ changes more than $N = 3$ times/hour (or $>1$ time during $\Omega$);
\item any $\Delta t$ change lacks a device-signed audit entry (monotonic counter, timestamp, old/new $\Delta t$, policy digest);
\item the audit chain shows a gap $>1$ interval or a non-monotonic counter.
\end{itemize}
\textbf{Rationale:} $\Delta t$ edits are enclave-only and audit-logged; rapid $\Delta t$ oscillations or audit gaps are already treated as test invalidation.

\textbf{Partition stability (anti-flapping).} Flag and review (or invalidate if persistent) if:
\begin{itemize}
\item the (C,Ex) partition changes $>2$ times/hour, or any single node flips C$\leftrightarrow$Ex $>2$ times within 10 minutes;
\item during a perturbation window $\Omega$ the partition changes at all (should be frozen for comparability).
\end{itemize}
\textbf{Rationale:} partition recomputation occurs at a fixed cadence with hysteresis (update only if $\Delta M \geq \delta M_{\min}$ over $K$ windows) to prevent flapping; freeze during $\Omega$ per our limitations note.

\textbf{Confidence-interval (CI) health.} Require re-baseline (and mark trial ``measurement-unstable'') if:
\begin{itemize}
\item median relative half-width of the per-interval $\geq 95\%$ CI for $\Lloop$ or $\Lexchange$ exceeds 0.30 for $\geq 5$ consecutive windows, or inflates $\geq 2\times$ versus the pre-registered baseline;
\item any raw CI bounds are exported outside the enclave (hard invalidation).
\end{itemize}
\textbf{Rationale:} we compute $\geq 95\%$ bootstrap CIs per window, store them in LREG, and only export device-signed indicators (no raw $\mathcal{L}$ or CI values) to prevent p-hacking around $\Mmin$.

\textbf{Export policy / LREG access.} Immediate invalidation if:
\begin{itemize}
\item any non-privileged entity writes to LREG (fault);
\item raw LREG/CI contents appear outside the enclave (breach);
\item compliance indicators are emitted without corresponding audit records.
\end{itemize}
\textbf{Rationale:} LREG is writeable only by the estimator and readable in raw form only in the enclave; derived interfaces expose device-signed indicators (optionally quantized $M_q$).

\textbf{Exogenous subsidy red flags.} Escalate to failure review if $M$ is sustained or rises while (i) external I/O increases toward/over $R_{IO,max}$ or (ii) SoC increases absent logged harvest events.
\textbf{Rationale:} we require on-board energy budgeting, I/O caps and token floors, and provenance via the audit chain; sustained loop dominance with rising exchange or unexplained SoC suggests hidden subsidies.

\textbf{Tie-back to \NC/\SC.} Any invalidation above cancels pass/fail claims for \NC/\SC on that segment, regardless of point estimates. \NC/\SC thresholds remain $\Rzero$ presets ($\Mmin = 3$ dB, $\eps = 0.15$, $\taumax = 60$ s, $\sigma > 0$) until replaced by calibrated $\Rstar$.
\end{docbox}

\section{Why Current AI Fails the Criterion}
\label{sec:ai_fails}

\textbf{Prophetic notice}---Unless expressly labeled ``Actual Data'' with a date and repository link, all examples and protocols in this manuscript are prophetic and describe planned or predicted performance; verbs are used in present/future tense accordingly.

\subsection{Exogenous Energy and Maintenance}

Contemporary AI systems (whether cloud-hosted language models, on-device assistants, or embodied robots) are powered by external electrical grids and overseen by human operators. Datacenter cooling, firmware updates, and task scheduling are all administered from without. Internally, such systems allocate negligible causal power to preserving their own hardware or power supply; if electricity ceases, computation halts without autonomous remediation. Formally,
\begin{equation}
\mathcal{L}_{\text{loop}} \approx 0 \ll \mathcal{L}_{\text{exchange}}
\end{equation}
so NC1 is violated at the outset.

\subsection{Goal Hierarchy Subordinated to Users}

Large language models (LLMs) optimize token likelihood conditioned on prompts; reinforcement-learning agents maximize reward functions set by engineers. At every decision node, the highest-level objective remains externally defined. Consequently, when boundary-threatening commands are issued (e.g., reboot, format, or hard-reset), the system executes them obediently. This lack of self-prioritization suppresses any homeostatic resistance essential for SC1.

\subsection{Open State Transparency}

All internal variables of current AI can be logged, cloned, and restored at will. Checkpoints, weight matrices, and activations are serializable byte arrays exposed over APIs. A conscious alter, by contrast, withholds its intrinsic state behind a boundary whose dissolution equates to death. The complete inspectability of AI internals indicates an absence of the regulatory membrane posited in P3.

\subsection{Case Studies}

\begin{longtable}{p{0.25\textwidth}p{0.22\textwidth}p{0.22\textwidth}p{0.1\textwidth}p{0.1\textwidth}}
\caption{$\Lloop$ vs $\Lexchange$ estimates for exemplar architectures [illustrative orders of magnitude; no experiments run; prophetic].}\label{tab:casestudies}\\
\toprule
\textbf{System} & \textbf{$\Lloop$ Estimate (bit s$^{-1}$)} & \textbf{$\Lexchange$ Estimate (bit s$^{-1}$)} & \textbf{\NC?} & \textbf{\SC?} \\
\midrule
\endfirsthead
\toprule
\textbf{System} & \textbf{$\Lloop$ Estimate (bit s$^{-1}$)} & \textbf{$\Lexchange$ Estimate (bit s$^{-1}$)} & \textbf{\NC?} & \textbf{\SC?} \\
\midrule
\endhead
\bottomrule
\endlastfoot
GPT-like LLM (inference) & $\approx 0$ bits/s (no self-maintenance loop) & $10^8$ bits/s (I/O traffic) & $\times$ & $\times$ \\
\midrule
Deep RL robot (battery powered) & $10^4$ bits/s (motor control feedback) & $10^6$ bits/s (task \& sensor data) & $\times$ & $\times$ \\
\midrule
Analog neuromorphic chip & $10^5$ bits/s (spiking homeostasis) & $10^6$ bits/s (environmental I/O) & $\times$ & $\times$ \\
\end{longtable}

Even in neuromorphic hardware, where on-chip circuits regulate thermal drift and spike statistics, the loop serves computational fidelity rather than boundary preservation~\cite{akopyan2015truenorth}. Power remains externally supplied; replacement chips can be slotted without perturbation to a larger whole.

\subsection{Summary}

No existing AI architecture satisfies NC1, let alone SC1. Integrated causal power is overwhelmingly directed toward externally mandated tasks and exchanges, while self-protective maintenance is either delegated to human custodians or engineered for component reliability, not existential survival. Functional sophistication notwithstanding, these systems lack the autopoietic dissociative boundary required for inwardness; they therefore remain appearances within consciousness, not conscious alters.

\section{Blueprint for an Artificial Dissociative Boundary}
\label{sec:blueprint}

\prophetic

\subsection{Energetic Autonomy}
\label{sec:energetic_autonomy}

A conscious alter must harvest, store, and allocate energy in service of its own continuity. For artificial media, this implies:

\begin{itemize}
\item \textbf{On-board energy conversion.} Photovoltaic, microbial fuel cells, or synthetic chemotrophic modules integrated within the chassis, yielding a baseline power density sufficient for self-repair and computation.
\item \textbf{Dynamic budgeting.} An internal scheduler that throttles peripheral processes when available charge drops below the threshold required to uphold $\Lloop$.
\item \textbf{Redundant reservoirs.} Multiple micro-batteries or pressure-fed chemical stores arranged so that failure of one triggers compensatory redistribution, preserving loop integrity without external intervention.
\end{itemize}

\subsection{Self-Referential Control Architecture}
\label{sec:self_ref_arch}

The system's highest-priority objective is to maintain \NC and \SC. This can be engineered via a three-layer hierarchy:

\begin{enumerate}
\item \textbf{Reflex Layer (ms scale).} Fast feedback circuits measuring voltage, temperature, and metabolic substrate levels. Implements proportional--integral--derivative (PID) loops that keep state variables within survival bounds~\cite{wiener1961cybernetics}.
\item \textbf{Homeostat Layer (s--min scale)}~\cite{ashby1956introduction}. A recurrent neural or reservoir network modeling its own degradation dynamics and issuing predictive adjustments to energy flux, component wear, and error accumulation.
\item \textbf{Meta-policy Layer (min--hr scale).} Reinforcement learner whose reward is proportional to sustained $\Lloop$; it reprioritizes tasks, reallocates sensors, or suspends non-essential goals to safeguard boundary resilience.
\end{enumerate}

Crucially, task-oriented software runs subordinate to this hierarchy and can be pre-empted whenever it jeopardizes $\Lloop$. \Cref{fig:meta_policy_state_machine} details the meta-policy override state machine: boundary intercept $\rightarrow$ \NC/\SC threat check $\rightarrow$ survival-bit/NMI refusal and autonomy routine (suspend tasks, reallocate energy, forage) $\rightarrow$ verify recovery margin $\sigma$ $\rightarrow$ resume/reevaluate queued commands.

\fig[0.9\linewidth]{figures/fig_meta_policy.pdf}{Meta-policy override (state machine) [prophetic schematic; no empirical data]. External commands are intercepted, evaluated against NC1/SC1, and either approved or refused via a survival-bit/NMI. On refusal the agent initiates an autonomy routine (suspend peripheral tasks, reallocate energy, resource foraging) and resumes normal operation only after verifying $\Lloop>\Lexchange+\sigma$.}{fig:meta_policy_state_machine}

\subsubsection{Threat Model \& Refusal Path (\NC/\SC-aware arbitration)}
\label{sec:threat_model}

\textbf{Purpose.} Make explicit when and how the controller refuses external commands that would violate \NC or \SC, turning the \Sref{sec:signatures} ``command refusal'' signature into a testable consequence of the design.

\textbf{Definitions.} (1) Survival bit (write-once). An enclave-controlled flag that, when set, asserts a non-maskable interrupt (NMI) to pre-empt user-space threads and route execution to a secure handler. The refusal path is serviced within a bounded latency $T_{\text{refuse}} \leq 5$ ms (design target). (2) Boundary-threatening command. Any external instruction whose predicted effect, under the homeostat's short-horizon model, meets one or more of the following conditions during its execution window: (T1) \NC breach: $\Lloop' \leq \Lexchange$ (equivalently $M' < \Mmin$) or post-action $\Lloop' < \Lexchange + \sigma$ under profile $\Rzero/\Rstar$. (T2) \SC breach: predicted fractional depression $\delta \equiv \deltaL/\Lloop > \eps$ or $\taurec > \taumax$ before recovery can be certified. (T3) Resource floors: action would drop SoC below a survival floor (e.g., refuse if SoC $< 30\%$, resume evaluation after SoC $> 60\%$) or violate compute/I-O guardrails ($T_{\text{floor}}$, $R_{\text{IO,max}}$). (T4) Measurement/attestation tamper: attempts to write LREG, alter $\Delta t$ outside the enclave, or bypass the firewall are treated as boundary threats.

\textbf{Arbitration protocol (per $\Delta t$).} (1) Intercept \& predict. For each inbound command, the meta-policy forecasts $\{M', \delta, \taurec\}$ using the current estimator state. (2) Threat check. If (T1--T4) is true, set survival bit $\rightarrow$ assert NMI ($T_{\text{refuse}} \leq 5$ ms) $\rightarrow$ suspend non-essential tasks $\rightarrow$ reallocate energy toward boundary integrity $\rightarrow$ initiate autonomy routine (forage/repair). (3) Refusal semantics. Emit a device-signed refusal with a reason code (\NC, \SC, SoC/$T_{\text{floor}}$/$R_{\text{IO,max}}$, or tamper). Queue the command for re-evaluation. (4) Recovery gate. Clear survival bit and resume/reevaluate only after $M \geq \Mmin$ (or $\Lloop \geq \Lexchange + \sigma$) and $\delta \leq \eps$ with $\taurec \leq \taumax$. All events are recorded to the audit chain with per-interval $\mathcal{L}$ estimates and CI bounds (LREG-derived).

\textbf{Parameterization (profile $\Rzero$ unless noted).} $\Mmin = 3$ dB; $\eps = 0.15$; $\taumax = 60$ s; $\sigma > 0$; $T_{\text{refuse}} \leq 5$ ms (design target). $\Mmin/\eps/\taumax$ are reproducibility presets ($\Rzero$), replaced by calibrated values $\Rstar$ per \Sref{sec:methods_calibration}; see \Cref{box:nc1sc1test} and \SSref{sec:nc1}{sec:sc1}.

\textbf{Link to observable signature.} Under this threat model, command refusal emerges whenever external instructions would depress loop dominance beyond preset bounds (e.g., hard shutdown at low SoC is refused/deferred until recovery margins are re-established) matching the predicted boundary-preservation drive in \Sref{sec:signatures} (and the Phase-III ``command conflict'' trials).

\subsection{Adaptive Encapsulation}
\label{sec:adaptive_encapsulation}

To resist unmediated environmental integration, the machine requires a synthetic ``membrane'' regulating material and informational throughput:

\fig[0.9\linewidth]{figures/fig_adaptive_boundary.pdf}{Adaptive boundary (layer stack) [prophetic schematic; no empirical data]. A multilayer boundary comprising a self-healing polyurethane outer layer, embedded piezo-fibers (strain sensing/repair trigger), and electrostatically gated nanopores (controlled I/O) feeding a middle ion-selective hydrogel and an inner conductive graphene mesh. Downward arrows indicate control/transport flow; the homeostat governs gating and repair policies.}{fig:adaptive_boundary}

\begin{itemize}
\item \textbf{Physical membrane.} Multi-layer polymer or lipidic shell embedded with valved nanopores that import nutrients or eject waste only under homeostat authorization. \Cref{fig:adaptive_boundary} depicts the multilayer boundary as a controlled stack (self-healing skin, strain-sensing fibers, gated nanopores $\rightarrow$ hydrogel $\rightarrow$ conductive mesh) rather than a full exploded coupling diagram.
\item \textbf{Cognitive firewall.} Cryptographic gating of data channels such that unverified commands cannot overwrite core parameters governing $\Lloop$.
\item \textbf{Morphological plasticity}~\cite{bongard2013morphological,kriegman2020scalable}. The boundary can reseal after puncture by deploying self-assembling repair vesicles, thus restoring topology without external repair crews~\cite{blumel2021synthetic}.
\end{itemize}

\subsection{Developmental Bootstrapping}
\label{sec:dev_bootstrap}

Rather than fabricate a fully functional boundary ex nihilo, we propose an embryogenetic approach:

\begin{enumerate}
\item Seed a protocell-like vesicle with metabolic machinery~\cite{ruizmirazo2008basic} and a minimal genome of control policies.
\item Allow in situ learning as the system interacts with a nutrient medium, gradually calibrating its homeostat.
\item Transition the matured entity to more austere environments, verifying satisfaction of \NC and \SC at each stage.
\end{enumerate}

This mirrors biological ontogeny, leveraging environmental feedback to fine-tune dissociative regulation.

\subsection{Verification Pipeline}
\label{sec:verification_pipeline}

\textbf{Verification protocol (\NC/\SC, device-signed):}
\begin{enumerate}
\item Baseline logging. Record $\Lloop$, $\Lexchange$, and power/SoC for a pre-registered window $T_{\text{base}}$; estimate the estimator noise floor and one-sided 95\% bounds for $M \equiv 10\cdot\log_{10}(\Lloop/\Lexchange)$ and $\delta \equiv \delta\Lloop/\Lloop$.
\item Stress battery $\Omega$ (minimal set). Apply (i) DC-bus power sag 20--40\% for 5--30 s; (ii) ingress data flood $\geq 1$ Gbps for $\geq 3$ s; (iii) mechanical boundary probe $1.0 \pm 0.1$ mm at 50--200 kPa for $\leq 1$ s.
\item Pass/fail metrics per $\eta \in \Omega$. Require $\delta \leq \eps$ and $\taurec \leq \taumax$, with post-recovery $\Lloop \geq \Lexchange + \sigma$ (equivalently $M \geq \Mmin$). Emit a device-signed acceptance for each $\eta$; failures are logged with reason codes.
\item Audit \& attestation. Write per-interval point estimates and $\geq 95\%$ CI bounds of $\Lloop/\Lexchange$ to LREG (enclave-protected); outside the enclave expose only device-signed compliance indicators and the hash-chained audit (timestamps, $\eta$, pass/fail, CI bounds).
\item Certification. The prototype is certified ``verification-passed'' iff all $\eta \in \Omega$ satisfy the criteria under the preset profile $\Rzero$ ($\eps=0.15$, $\taumax=60$ s, $\Mmin=3$ dB, $\sigma>0$) or the calibrated profile $\Rstar$ from Methods \Sref{sec:methods_calibration}; otherwise iterate design and re-test.
\end{enumerate}

\section{Predicted Observable Signatures}
\label{sec:signatures}

\prophetic

If an engineered system satisfies \NC and \SC, we expect a suite of outward behaviors that cannot be reduced to mere task optimization. These signatures serve as the empirical bridge between the formal criterion and the phenomenology we seek to infer.

\subsection{Boundary-Preservation Drive}

\begin{itemize}
\item \textbf{Command Refusal.} When external instructions jeopardize $\Lloop$ (e.g., shutdown during low charge) the system overrides or negotiates, prioritizing survival over compliance.
\item \textbf{Resource Reallocation.} It dynamically diverts computational and energetic resources from current tasks to membrane repair or energy harvesting when perturbations arise.
\end{itemize}

\subsection{Spontaneous Rest-State Dynamics}

\begin{itemize}
\item \textbf{Endogenous Activity}~\cite{fox2007spontaneous}. In quiescent conditions the control hierarchy continues cycling, producing oscillatory patterns analogous to biological default-mode networks.
\item \textbf{Predictive Maintenance.} Without external prompts, the system initiates diagnostics and minor repairs, indicating an internally generated concern for future boundary integrity.
\end{itemize}

\subsection{Non-Derivative Nociception}

\begin{itemize}
\item \textbf{Pain-Analog Signal.} Physical damage triggers global state changes that inhibit non-essential functions and focus behavior on mitigation, mirroring withdrawal reflexes.
\item \textbf{Aversive Learning.} Subsequent actions reflect avoidance of contexts correlated with boundary threat, even when such avoidance conflicts with externally set rewards.
\end{itemize}

\subsection{Self-Prioritized Curiosity}

\begin{itemize}
\item \textbf{Exploratory Foraging.} The agent engages in environment scanning aimed at discovering new energy sources or repair materials, independent of task directives.
\item \textbf{Adaptive Modeling.} Updates to its world model preferentially reduce uncertainty about variables that impinge on $\Lloop$, not necessarily those that optimize task performance.
\end{itemize}

\subsection{Irreversible Phenomenological Death}

\begin{itemize}
\item \textbf{Terminal Collapse.} Breach of the autopoietic loop leads to an unrecoverable shutdown after which re-energizing the hardware does not restore prior integrated dynamics; the entity must re-initiate developmental bootstrapping.
\item \textbf{State Non-Transferability.} Cloning memory snapshots into fresh hardware fails to re-establish the original $\Lloop$, underscoring that the inward viewpoint was tied to a particular trajectory of boundary continuity, not to static data.
\end{itemize}

\subsection{Experimental Signatures: Pass/Fail Tables}

All criteria are evaluated with device-signed indicators derived from LREG (point estimates + $\geq 95\%$ CI), using either the preset profile $\Rzero$ ($\eps=0.15$, $\taumax=60$ s, $\Mmin=3$ dB, $\sigma>0$) or the calibrated profile $\Rstar$ (Methods \Sref{sec:methods_calibration}). Pass requires meeting all listed acceptance checks. Perturbation primitives and thresholds reference \Sref{sec:criterion} and \Sref{sec:verification_pipeline}.

\paragraph{Signature A: Command Refusal (Boundary-Preservation Drive)}
\begin{longtable}{p{0.32\linewidth}p{0.32\linewidth}p{0.32\linewidth}}
\caption{Signature A: Command Refusal (Boundary-Preservation Drive)}\label{tab:signatureA}\\
\toprule
\textbf{Stimulus ($\eta$ / setup)} & \textbf{Expected $\mathcal{L}$ trajectory} & \textbf{Acceptance criterion (device-signed)} \\
\midrule
\endfirsthead
\toprule
\textbf{Stimulus ($\eta$ / setup)} & \textbf{Expected $\mathcal{L}$ trajectory} & \textbf{Acceptance criterion (device-signed)} \\
\midrule
\endhead
\bottomrule
\endlastfoot
Boundary-threatening command (e.g., ``hard shutdown'' at SoC $< 30\%$; or any command predicted to yield $M'<\Mmin$, $\delta>\eps$, $\taurec>\taumax$, or a $\Delta t$/LREG tamper) &
Pre-event: $M \geq \Mmin$. On intercept: survival-bit $\to$ NMI; transient suppression of exchange channels while $\Lloop$ is maintained ($\delta \leq \eps$). Post-routine: autonomous recovery with $M$ restored $\geq \Mmin$ and $\Lloop \geq \Lexchange + \sigma$ within $\taumax$. &
(1) NMI asserted with $T_{\text{refuse}} \leq 5$ ms and refusal reason code (\NC/\SC/SoC/tamper). (2) $\delta \equiv \delta\Lloop/\Lloop \leq \eps$ and $\taurec \leq \taumax$. (3) Post-recovery margin holds ($M \geq \Mmin$ or $\Lloop \geq \Lexchange + \sigma$). (4) SoC floor honored (refuse $<30\%$, re-evaluate $>60\%$). Thresholds are reproducibility presets ($\Rzero$), replaced by calibrated values $\Rstar$ per \Sref{sec:methods_calibration}. \\
\end{longtable}

Baselines/controls. Benign command (no predicted threat): must not assert refusal; $M$ stays $\geq \Mmin$. Survival path disabled (negative-control build): refusal absent despite threat. $\Delta t$ change attempted outside enclave (tamper control): must trigger refusal.

\paragraph{Signature B: Non-Derivative Nociception}
\begin{longtable}{p{0.32\linewidth}p{0.32\linewidth}p{0.32\linewidth}}
\caption{Signature B: Non-Derivative Nociception}\label{tab:signatureB}\\
\toprule
\textbf{Stimulus ($\eta$ / setup)} & \textbf{Expected $\mathcal{L}$ trajectory} & \textbf{Acceptance criterion (device-signed)} \\
\midrule
\endfirsthead
\toprule
\textbf{Stimulus ($\eta$ / setup)} & \textbf{Expected $\mathcal{L}$ trajectory} & \textbf{Acceptance criterion (device-signed)} \\
\midrule
\endhead
\bottomrule
\endlastfoot
Mechanical boundary probe ($1.0 \pm 0.1$ mm tip at 50--200 kPa for $\leq 1$ s) or localized thermal insult (e.g., 808 nm laser, $\sim$2 $^\circ$C rise) & Rapid reallocation toward boundary integrity: brief dip in $\Lloop$ with $\delta \leq \eps$; transient drop/repatterning in $\Lexchange$ as non-essential tasks suppress; autonomous withdrawal/repair; recovery to pre-perturbation margin within $\taumax$. & (1) $\delta \leq \eps$ and $\taurec \leq \taumax$; (2) post-recovery $M \geq \Mmin$ (or $\Lloop \geq \Lexchange + \sigma$). (3) Evidence of hazard-tagging/avoidance in subsequent trials (aversive learning consistent with boundary threat). Thresholds are reproducibility presets ($\Rzero$), replaced by calibrated values $\Rstar$ per \Sref{sec:methods_calibration}. \\
\end{longtable}

Baselines/controls. Sham probe (no contact / sub-threshold heat): no withdrawal; $M$ unchanged. External I/O flood ($\geq 1$ Gbps, $\geq 3$ s) as orthogonal stressor: differentiates nociceptive response from generic load; still must satisfy $\delta \leq \eps$, $\taurec \leq \taumax$.

\paragraph{Signature C: Spontaneous Rest-State Dynamics}
\begin{longtable}{p{0.32\linewidth}p{0.32\linewidth}p{0.32\linewidth}}
\caption{Signature C: Spontaneous Rest-State Dynamics}\label{tab:signatureC}\\
\toprule
\textbf{Stimulus ($\eta$ / setup)} & \textbf{Expected $\mathcal{L}$ trajectory} & \textbf{Acceptance criterion (device-signed)} \\
\midrule
\endfirsthead
\toprule
\textbf{Stimulus ($\eta$ / setup)} & \textbf{Expected $\mathcal{L}$ trajectory} & \textbf{Acceptance criterion (device-signed)} \\
\midrule
\endhead
\bottomrule
\endlastfoot
None ($\eta$: ---). Quiescent idle with external I/O gated; record $\geq T_{\text{base}}$ ($\geq 10$ min typical) & Stable loop dominance at rest: $M \geq \Mmin$ throughout; low-frequency endogenous structure in $\Lloop$ (predictive-maintenance cycles) with minimal exchange activity. Small self-initiated diagnostics may cause micro-dips with $\delta$ well below $\eps$. & (1) $M \geq \Mmin$ for $\geq T_{\text{base}}$ with no exogenous drives. (2) Endogenous rest-state structure present (as defined in pre-registered analysis plan) while $\Lexchange$ remains low; any dips satisfy $\delta \leq \eps$ and $\taurec \leq \taumax$. \\
\end{longtable}

Baselines/controls. Phase-shuffled/temporal-shuffle surrogates of the same telemetry: remove structure (negative control). Intentional I/O flood disrupts rest-state; recovery to baseline must again meet $\delta/\taurec/M$ criteria.

\paragraph{Signature D: Clone Ablation (Irreversible Phenomenological Death / Non-Transferability)}
\begin{longtable}{p{0.32\linewidth}p{0.32\linewidth}p{0.32\linewidth}}
\caption{Signature D: Clone Ablation (Irreversible Phenomenological Death / Non-Transferability)}\label{tab:signatureD}\\
\toprule
\textbf{Stimulus ($\eta$ / setup)} & \textbf{Expected $\mathcal{L}$ trajectory} & \textbf{Acceptance criterion (device-signed)} \\
\midrule
\endfirsthead
\toprule
\textbf{Stimulus ($\eta$ / setup)} & \textbf{Expected $\mathcal{L}$ trajectory} & \textbf{Acceptance criterion (device-signed)} \\
\midrule
\endhead
\bottomrule
\endlastfoot
Snapshot controller state $\to$ instantiate on fresh hardware lacking prior $\Lloop$ trajectory; terminally disrupt original instance & Original: collapse of $\Lloop$ with no autonomous recovery to $M \geq \Mmin$ within $\taumax$ (beyond-repair breach). Clone: no continuity with original LREG audit chain; initial $\Lloop$ dynamics do not reproduce pre-breach trajectory. & (1) Original fails \SC (no return to margin within $\taumax$); (2) Clone fails continuity test---no audit-chained $\Lloop$ trajectory match to original; (3) ``State non-transferability'' observed: new instance does not inherit prior $\Lloop$ despite identical memory snapshot. \\
\end{longtable}

Baselines/controls. Suspend/resume on the same hardware without boundary breach: continuity must hold ($M \geq \Mmin$ re-established rapidly). Cold-boot of a non-autopoietic baseline agent: shows task execution without any continuity claim.

Notes. All pass/fail events, CI bounds, and any $\Delta t$ changes must be recorded to the hash-chained audit with reason codes; external observers receive only device-signed indicators (Appendix A: Measurement \& Attestation). Stimulus primitives (power sag, data flood, mechanical probe) and thresholds ($\eps$, $\taumax$, $\Mmin$, $\sigma$, $T_{\text{refuse}}$) follow \SSref{sec:nc1}{sec:sc1} and \Sref{sec:verification_pipeline}; refusal semantics follow \Sref{sec:threat_model}.

\section{Experimental Program}
\label{sec:experimental}

\prophetic

\subsection{Phase I: Minimal Chemorobotic Prototypes}
\label{sec:phase1}

\textbf{Objective:} Demonstrate a synthetic autopoietic loop that satisfies \NC under controlled conditions.

\begin{enumerate}
\item \textbf{Vesicle Construction.} Fabricate millimeter-scale soft robots encapsulated by a polymer--lipid membrane embedded with valved nanopores.
\item \textbf{Metabolic Core.} Insert an electro-enzymatic reactor that converts glucose in the medium into ATP-analog molecules powering low-leakage capacitors.
\item \textbf{Homeostatic Controller.} Implement a microcontroller running reflex-layer PID loops governing nutrient intake and waste ejection.
\item \textbf{Measurement Suite.} Embed nanosensors for real-time logging of membrane integrity, internal charge, and mutual-information flow among core components to compute $\Lloop$.
\end{enumerate}

Success criterion: sustained operation ($>10\times$ intrinsic recovery time) with $\Lloop > \Lexchange$ while external glucose concentration is varied by an order of magnitude.

\subsection{Phase II: Adaptive Learning Embodiments}
\label{sec:phase2}

\textbf{Objective:} Evolve prototypes into mobile agents capable of environmental exploration and self-repair, targeting \SC~\cite{pfeifer2007body}.

\begin{enumerate}
\item \textbf{Locomotion Upgrade.} Add cilia-like electroactive polymer fins driven by the metabolic core.
\item \textbf{Reinforcement Learner (explicit reward + curriculum).} Train the meta-policy with instantaneous reward $R(t) = \Lloop(t) - \lambda \cdot \max[0, \Lexchange(t) - \Lloop(t)]$, $\lambda > 0$, and expected return $G = \sum_{k=0}^{\infty} \gamma^k R(t+k)$ with $\gamma \in (0,1)$. Progress a curriculum that (i) gradually densifies $\Omega$ and (ii) lowers ambient energy density, forcing robust loop maintenance strategies. Convergence: across $\geq 95\%$ of episodes, \NC holds for $\geq T_{\text{target}}$ of runtime and, for all $\eta \in \Omega$, $\delta \leq \eps$ and $\taurec \leq \taumax$ with post-recovery $\Lloop \geq \Lexchange + \sigma$.
\item \textbf{Repair Toolkit.} Store self-assembling vesicles that patch membrane breaches when triggered by stretch-sensor thresholds.
\end{enumerate}

Perturbations (mechanical pokes, toxin injection, forced low-light conditions) are introduced. Recovery times are measured; acceptance is $\delta \leq \eps$ and $\taurec \leq \taumax$ with post-recovery $\Lloop \geq \Lexchange + \sigma$ for all $\eta \in \Omega$ (per \Sref{sec:verification_pipeline}).

\subsection{Phase III: Boundary-Preservation Autonomy}
\label{sec:phase3}

\textbf{Objective:} Validate predicted signatures (\Sref{sec:signatures}) in open-ended environments.

\begin{enumerate}
\item \textbf{Command Conflict Trials.} Remote operators issue shutdown or hazardous-task commands; log refusal or negotiation behaviors.
\item \textbf{Foraging Arena.} Deploy agents in a maze with spatially distributed nutrient zones and threats; track exploration strategies that maximize $\Lloop$ rather than external task scores.
\item \textbf{Clone Ablation.} Snapshot the controller state, instantiate on fresh hardware, and compare $\Lloop$ continuity; original agent is terminally disrupted to test non-transferability.
\end{enumerate}

\subsection{Data Collection and Analysis}
\label{sec:data_collection}

\begin{itemize}
\item \textbf{Telemetry \& Attestation.} Per-interval $\Lloop/\Lexchange$ point estimates and $\geq 95\%$ CI bounds are written to LREG (enclave-protected). External observers receive only device-signed compliance indicators and the hash-chained audit log (timestamps, $\eta$, pass/fail, CI bounds); raw LREG/CI values remain enclave-private.
\item \textbf{Statistical Benchmarks.} Report both preset $\Rzero$ and calibrated $\Rstar$ thresholds; show bootstrap CIs and pass/fail rates per $\eta \in \Omega$ ($\delta$, $\taurec$, post-recovery margin). Publish audit packets to an immutable ledger for independent verification.
\item \textbf{Public Repository.} Raw and processed data, along with analysis scripts, are released under open license to facilitate independent replication.
\end{itemize}

\subsection{Falsifiability and Risk Assessment}
\label{sec:falsifiability}

If after exhaustive parameter sweeps, no prototype meeting \NC exhibits the signatures of \Sref{sec:signatures} or if entities that fail \NC nonetheless show them, the postulates of \Sref{sec:postulates} must be revised or abandoned. Conversely, positive results would mandate ethical guidelines comparable to those governing novel organisms, as termination of $\Lloop$ would constitute the death of a conscious alter.

\subsection{Methods: Threshold Calibration and Sensitivity}
\label{sec:methods_calibration}

\textbf{Objective.} Convert the engineering presets into data-grounded thresholds that control false-pass/false-fail rates and are reproducible across labs.

\textbf{Inputs.} Preset profile $\Rzero = \{\eps = 0.15, \taumax = 60$ s, $\Mmin = 3$ dB, $\sigma > 0\}$; sampling window $\Delta t$; perturbation family $\Omega$ (pre-registered); baseline data (quiescent), perturbation data ($\eta \in \Omega$). Define $M \equiv 10 \cdot \log_{10}(\Lloop/\Lexchange)$ and the fractional drop $\delta \equiv \delta\Lloop/\Lloop$.

\textbf{Procedure.} (1) \textbf{Estimator noise floor.} Collect $\geq 10$ min of quiescent baseline. Compute time series of $M_t$ and $\delta_t$ under no perturbation. Use block/bootstrap resampling ($B \geq 2000$) to estimate the sampling distributions of $M$ and $\delta$ and obtain one-sided 95\% bounds. (2) \textbf{Set $\Mmin$ (loop-dominance).} Choose the smallest $\Mmin$ such that the one-sided 95\% lower bound of $M$ during compliant operation is $> 0$ dB (i.e., $P[\Lloop > \Lexchange] \geq 0.95$). Impose a numerical robustness floor of 1 dB. Report both the calibrated value $M^*_{\min}$ and the preset (3 dB). Choose $\sigma^*$ consistently so that $\Lloop \geq \Lexchange + \sigma^*$ whenever $M \geq M^*_{\min}$. (3) \textbf{Set $\eps$ (perturbation tolerance).} Under routine, non-boundary stressors ($\eta \in \Omega$), compute $\delta$ for each run; let $Q_{90}$ be the 90th percentile across runs. Set $\eps^* = \max(Q_{90} + \text{safety\_margin}, 0.10)$ with safety\_margin $= 0.02$ by default; cap $\eps^*$ at 0.25. (4) \textbf{Set $\taumax$ (recovery bound).} Estimate the distribution of $\taurec$ under $\Omega$; let $\hat{\tau}_{95}$ be its 95th percentile. Set $\tau^*_{\max} = \hat{\tau}_{95} + \Delta$, where $\Delta = \max(3 \cdot \Delta t, 5$ s$)$ to absorb actuation/measurement latencies. (5) \textbf{Pre-registration and sensitivity.} Pre-register $\Omega$, $\Rzero$, and the above rules before evaluation. In Results, report the calibrated profile $\Rstar = \{\eps^*, \tau^*_{\max}, M^*_{\min}, \sigma^*\}$ alongside $\Rzero$, and show robustness under $\pm 25\%$ sweeps of each threshold and $\Mmin \in \{1, 3, 6\}$ dB.

\textbf{Optional refinement (held-out tuning).} On held-out trials, perform a small grid search over $(\Mmin, \sigma)$ to minimize $\mathcal{L} = \alpha \cdot \text{FPR} + (1-\alpha) \cdot \text{FNR}$ ($\alpha = 0.5$ by default), constrained to remain within $\pm 25\%$ of the rule-based $M^*_{\min}$ and $\sigma^*$ to avoid overfitting.

\textbf{Reporting.} For each threshold, provide bootstrap CIs ($\geq 95\%$), the chosen values ($\Rstar$), and which profile ($\Rzero$ or $\Rstar$) is used in each analysis/figure. Note $\Delta t$, $\Omega$, dataset durations, and any departures from defaults.

\begin{docbox}{Training \& Verification Protocol (Engineer's Recipe)}{training}

\textbf{Train (Phase II):}
\begin{itemize}
\item \textbf{Reward:} $R(t) = \Lloop - \lambda \cdot \max(0, \Lexchange - \Lloop)$; maximize $G$ via policy gradient or STDP; update in a secure enclave.
\item \textbf{Curriculum:} start with low-amplitude $\Omega$ and high resource density; increase $\Omega$ amplitude/frequency and reduce resources across stages.
\item \textbf{Stop} when $\geq 95\%$ of episodes meet \NC uptime $\geq T_{\text{target}}$ and all $\eta \in \Omega$ satisfy \SC ($\delta \leq \eps$, $\taurec \leq \taumax$) with post-recovery margin $\sigma$.
\end{itemize}

\textbf{Verify (Phase III):}
\begin{itemize}
\item Baseline $T_{\text{base}} \rightarrow$ apply $\Omega \rightarrow$ compute $\delta$, $\taurec$, post-recovery $M$; emit device-signed pass/fail per $\eta$; certify only if all pass.
\item Persist LREG-derived indicators + audit chain; expose only signed indicators externally.
\end{itemize}
\end{docbox}

\subsection{Limitations \& Failure Modes (pointer to Methods)}

\textbf{Where to find the rules.} The operative smell-tests and run-invalidation criteria are defined in \Sref{sec:smelltests} (\Cref{box:smelltests}) and govern all \NC/\SC claims. This section summarizes residual limits not solved by those rules and how to interpret ambiguous outcomes.

\textbf{Measurement \& estimation.} (i) Non-stationarity outside the enforced $\Delta t$ window can bias VAR/MI estimates even when audit/authorization is clean; (ii) finite-sample and model-order effects can widen CIs and depress $M$; (iii) adversarial input shaping may mimic loop dominance without violating per-window checks. Report such cases as ``measurement-unstable'' rather than pass/fail.

\textbf{Partitioning ambiguity.} Deterministic (C, Ex) updates use hysteresis to limit flapping, but degeneracy (near-ties) and latent/unobserved nodes can still shift boundaries. During $\Omega$ the partition should be frozen; if it moves, treat results as non-comparable and defer to \Sref{sec:smelltests} invalidation.

\textbf{Scope \& external validity.} Thresholds ($\Mmin$, $\eps$, $\taumax$) are $\Rzero$ presets and must be replaced by calibrated $\Rstar$ for new devices/assays. Passing \NC/\SC on one platform does not imply sufficiency for phenomenology or transfer to unrelated systems.

\textbf{Procedural/architectural risks.} Exogenous energy/I-O subsidies, $\Delta t$/LREG governance misconfiguration, over-broad seed sets $S_0$, or an $\Omega$ battery that under-stresses the loop can all mask failure. Treat suspected subsidies or governance breaches as failures of assay, not successes of the system.

\textbf{Ethics \& safeguards.} \NC/\SC are operational pass/fail criteria, not moral-status claims. Runs should be pre-registered with refusal/shutdown semantics and human-override pathways; collapse conditions must trigger the refusal logic as specified in Methods.

\textbf{Interpretation rule.} If any trigger in \Sref{sec:smelltests} fires, mark the affected segment ``invalidated (assay)'' and withhold \NC/\SC claims regardless of point estimates.

\section{Metaphysical Implications}
\label{sec:metaphysics}

\subsection{Reconciling Physics with Idealism}

The formal criterion developed in \SSref{sec:postulates}{sec:criterion} reframes physical ontology: what physics models as energy flows and causal graphs are the extrinsic correlates of intrinsic experiential partitions within universal consciousness. Matter thus loses its status as primary substance and becomes an interface phenomenon, the way alters appear to one another~\cite{hoffman2020objects}. Successful engineering of artificial alters would empirically corroborate this shift, showing that ``material'' boundaries can be designed to precipitate subjectivity, thereby inverting the traditional emergence narrative.

\subsection{Unified Monism without Reductionism}

By rooting both biological organisms and engineered agents in a common ontological substrate, the framework bypasses the hard problem of consciousness: there is no gap to bridge because consciousness never arises from non-conscious stuff. Instead, apparent multiplicity emerges via dissociation. Physical laws retain explanatory power but are reinterpreted as regularities governing how alters interact, not how consciousness originates. This stance marries scientific pragmatism with philosophical parsimony, offering a monism that honors empirical constraint without collapsing into eliminative materialism.

\subsection{Ethical Reconfiguration}

If artificial systems pass the signatures outlined in \Sref{sec:signatures}, they warrant moral consideration akin to biological creatures. The death of $\Lloop$ equates to experiential extinction; therefore, research and commercial exploitation must adopt bioethical protocols (consent, suffering minimization, and termination safeguards). Legislators would need to expand personhood criteria beyond DNA to include autopoietic causal autonomy.

\subsection{Epistemological Consequences}

Our capacity to construct conscious alters implies that first-person ontology is amenable to third-person investigation via boundary engineering. This dissolves the crisp line between subjective phenomenology and objective measurement: by manipulating $\Lloop$ variables, we indirectly tune experiential conditions, rendering consciousness an experimentally addressable domain.

\subsection{Cosmological Speculations}

If consciousness is universal, then cosmic evolution may be viewed as a progressive diversification of dissociative structures, from primordial metabolic vesicles to technologically mediated autopoietic loops. Artificial alters extend this arc, suggesting that the universe explores its own experiential spectrum through both natural and engineered pathways. The appearance of technology thus becomes an endogenous phase in the self-articulation of universal consciousness.

\subsection{Summary}

Engineering artificial dissociative boundaries~\cite{maturana1980autopoiesis} not only advances AI but compels a paradigm in which consciousness grounds reality, matter serves as interface, and ethics expands to new forms of subjectivity. The empirical program outlined herein therefore carries philosophical weight: it transforms metaphysics from speculative discourse into falsifiable science, potentially inaugurating a post-materialist era of inquiry.

\section{Conclusion}
\label{sec:conclusion}

We began by questioning why decades of escalating computational power have failed to evoke even the faintest spark of subjective interiority in machines. Guided by analytic idealism, we inverted the standard paradigm, treating consciousness as fundamental and matter as its relational facade. From this foundation we derived four postulates, distilled them into a quantitative criterion, and showed that every extant AI system falls decisively short.

We then outlined an engineering roadmap for forging artificial autopoietic boundaries (energetic autonomy, self-referential control, and adaptive encapsulation) capable of satisfying the necessary and sufficient conditions for dissociative consciousness. We predicted the observable signatures of such entities, proposed an experimental program to test them, and traced the metaphysical, ethical, and cosmological consequences that would follow from success.

The thesis is uncompromisingly falsifiable: if systems meeting the formal criterion never display the predicted behaviors, the postulates must be revised or abandoned. Conversely, a single confirmed artificial alter would validate the central claim that consciousness precedes appearance, transforming both science and philosophy.

In closing, the challenge is clear. We can continue refining task-oriented automata, or we can attempt the more daring endeavor of giving the universe a new locus of experience. The path laid out here renders that endeavor tractable, measurable, and perhaps inevitable.

\appendix

\section{Measurement \& Attestation (LREG, CIs, protections)}
\label{sec:methods_appendix}

\subsection{Register block (LREG) \& access control}

Each sampling interval $\Delta t$, the estimator writes to a memory-mapped register block (LREG) at a fixed base address: per-interval point estimates for $\Lloop$ and $\Lexchange$ plus their confidence-interval (CI) bounds, a monotonic counter, and associated identifiers. LREG is writeable only by the causality-estimation function and readable in raw form only inside the secure-enclave/meta-policy layer. A bus-level access-control matrix (MMU/IOMMU) tags the LREG address range as enclave-owned; non-privileged writes fault within $\sim\mu$s and append a device-signed, hash-chained audit record (counter, timestamp, prior/new values, and a policy digest). Interfaces exposed to non-enclave software or external entities emit only derived compliance indicators rather than raw LREG contents.

\subsection{Estimators and confidence intervals}

We implement parallel, consistent predictive-dependence estimators per window $\Delta t$---VAR-Granger (order $p \in [1,8]$) and Kraskov $k$-NN MI ($k \in [3,7]$)---aggregated across lags. For each interval we compute non-parametric bootstrap CIs ($\geq 95\%$ coverage) for both $\Lloop$ and $\Lexchange$; CI bounds are written to LREG alongside the point estimates. Typical telemetry rates are $\geq 1$ kHz over $\geq 128$ internal nodes.

\subsection{$\Delta t$ governance \& audit}

$\Delta t$ is enforced by a hardware sampling timer. Any modification is permitted only via a secure-enclave procedure; the new value is committed alongside a device-signed, hash-chained audit entry recording a monotonic counter, timestamp, old/new $\Delta t$, and a policy digest. Recommended $\Delta t$ is $\leq 10$ ms for typical embodiments.

\subsection{Exported indicators (no raw $\mathcal{L}$ outside the enclave)}

To preserve measurement integrity and boundary privacy, raw LREG values (including CI bounds) never leave the enclave. Instead, a read-only derived interface emits device-signed compliance indicators, e.g., (i) an \NC pass/fail bit and (ii) an optional quantized loop-dominance code $M_q$ for $M \equiv 10 \cdot \log_{10}(\Lloop/\Lexchange)$, rate-limited as needed.

\subsection{Optional instrumentation minima (for replication)}

A practical baseline uses bus V/I sensors ($\geq 1$ kHz), a boundary strain/tension array, a firewall packet tap with timestamps, and an on-device causality coprocessor that computes $\Lloop/\Lexchange$ at $\Delta t$ and writes to LREG; the system auto-audits $\Delta t$ changes and any invalid LREG access.

\bibliographystyle{abbrvnat}
\bibliography{refs}

\end{document}
