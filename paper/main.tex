\documentclass[11pt]{article}
\usepackage[margin=1in]{geometry}
\usepackage{amsmath,amssymb,mathtools}
\usepackage{graphicx}
\usepackage{hyperref}
\usepackage{cleveref}
\usepackage[numbers]{natbib}

% Centralized notation and environments

% Sets and operators
\newcommand{\R}{\mathbb{R}}
\DeclareMathOperator*{\argmin}{arg\,min}
\DeclareMathOperator*{\argmax}{arg\,max}

% Common symbols
\newcommand{\E}{\mathbb{E}}
\newcommand{\Var}{\mathrm{Var}}
\newcommand{\sys}{LDTC}

% Provide \gitversion from version.tex if present, otherwise default to "dev"
\IfFileExists{version.tex}{\input{version}}{\providecommand{\gitversion}{dev}}

% ----------------------------------------------------------------------------
% Metadata
% ----------------------------------------------------------------------------
\title{LDTC: A Reproducible Boundary Organism for NC1/SC1 Verification}
\author{Your Name}
\date{\today\ (\gitversion)}

\begin{document}
\maketitle

\begin{abstract}
This is a placeholder abstract. Replace with a concise summary of contributions and findings.
\end{abstract}

\section{Introduction}
This scaffold places the manuscript next to the code so figures and results can be regenerated deterministically. See \cref{fig:hello} for a placeholder figure and \citet{knuth1984texbook} for a demo citation.

\section{Example Figure}
\begin{figure}[t]
  \centering
  \includegraphics[width=0.7\linewidth]{figures/fig_hello.pdf}
  \caption{Placeholder figure generated by \texttt{paper/scripts/make\_fig\_hello.py}.}
  \label{fig:hello}
\end{figure}

\section{Conclusion}
Replace this scaffold with your actual content. Keep notation in `macros.tex`, add references in `refs.bib`, and generate figures via `paper/scripts/` invoked by the Makefile.

\bibliographystyle{abbrvnat}
\bibliography{refs}

\end{document}
